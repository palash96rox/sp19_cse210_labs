\documentclass{article}
\usepackage[utf8]{inputenc}
\usepackage{listings}
\usepackage{algorithm}
\usepackage{algorithmic}

\title{cse210 ass01}
\author{pa5795}
\date{2 April 2019}

\begin{document}

\maketitle

\section{Master's Theorem}
Extended master's theorem: $T(n) = a T(\frac{n}{b}) + \Theta(n^k log^p n)$
\subsection{Case 1: $ a > b^k \Rightarrow T(n) = \Theta(n^{log_b a}) $}
\subsection{Case 2: $ a = b^k $}
\subsubsection{$ p > -1 \Rightarrow T(n) = \Theta(n^k log^{p+1} n)$}
\subsubsection{$ p = -1 \Rightarrow T(n) = \Theta(n^k log(log n))$}
\subsubsection{$ p < -1 \Rightarrow T(n) = \Theta(n^k)$}
\subsection{Case 3: $ a < b^k $}
\subsubsection{$ p \geq 0 \Rightarrow T(n) = \Theta(n^k log^p n)$}
\subsubsection{$ p < 0 \Rightarrow T(n) = O(n^k)$}


\section{Minimum Spanning Trees}
\subsection{Prim's Algorithm}
Time Complexity: $O((V+E) log(V))$
\begin{algorithm}
\begin{algorithmic}
\STATE $ V \leftarrow$ (number of vertices)
\STATE $ U \leftarrow 1 $
\STATE $ T \leftarrow \emptyset $
\WHILE{$ V \ne U $}
\STATE $ (u,v) \leftarrow$ Lowest cost edge; $u \in U$ and $v \in V-U $
\STATE $ T \leftarrow T \cup (u,v) $
\STATE $ U \leftarrow U \cup {v} $
\ENDWHILE
\end{algorithmic}
\end{algorithm}
\subsection{Kruskal's Algorithm}
Time Complexity: $O(E log(V))$
\begin{algorithm}
\begin{algorithmic}
\STATE $ V \leftarrow$ (number of vertices)
\STATE $ T \leftarrow \emptyset $
\FOR{e in G.E (Ordered by weight in ascending order)}
\IF{$find(u) \neq find(v)$}
\STATE $T = T \cup (u,v)$
\STATE Union(u,v)
\ENDIF
\ENDFOR
\end{algorithmic}
\end{algorithm}


\section{Activity Selection}
\begin{algorithm}
\begin{algorithmic} 
\REQUIRE lists sorted in ascending order by end times
\STATE $i \leftarrow 0$
\STATE $j \leftarrow 1$
\STATE $activities = []$
\WHILE{$j < N$}
\IF{$start[j] > end[i]$}
\STATE $activities[i] \leftarrow (start[j],end[i])$
\STATE $i = j$
\ENDIF
\STATE $j \leftarrow j+1$
\ENDWHILE
\end{algorithmic}
\end{algorithm}
Time Complexity: As sorted array is required by the algorithm, the time complexity is just $O(n)$.


\section{Time Table Scheduler}
Code[Python]:
\begin{lstlisting}[language=python]
def time_table(start,end,k):
    n = len(start)
    arr = sorted(list(zip(start,finish)),key=lambda x: x[1])
    rooms = [[] for _ in range(k)]
    for i in range(n):
        for j in rooms:
            if not j or j[-1]<=arr[i][0]:
                j.append(arr[i][j])
                break
    return rooms
\end{lstlisting}


\section{Asymptotic Notations}

\subsection{Big-O: Upper Bound}
This gives the upper bound of a function. If $f(n) = O(g(n)$, the time complexity of the function ($f(n)$) is at least $g(n)$.\\
\textbf{Rule}: $f(n) = O(g(n))$ iff $\exists c, x_0$, such that, $f(x) \leq c g(x), x \geq x_0$\\
\textbf{Implication}: The growth rate of $f(n) \leq$ the growth rate of $g(n)$\\
\textbf{Example}: $f(2n+3) = O(g(10n)) = O(g(n)) = O(g(n^2)) = O(g(n^3)) = ...$

\subsection{Omega $\Omega$: Lower Bound}
This gives the lower bound of a function. If $f(n) = \Omega(g(n)$, the time complexity of the function ($f(n)$) is at most $g(n)$.\\
\textbf{Rule}: $f(n) = \Omega(g(n))$ iff $\exists c, x_0$, such that, $f(x) \geq c g(x), x \geq x_0$\\
\textbf{Implication}: The growth rate of $f(n) \geq$ the growth rate of $g(n)$\\
\textbf{Example}: $f(n^3) = \Omega(g(n^2)) = \Omega(g(n)) = \Omega(g(logn)) = ...$

\subsection{Theta $\Theta$: Range Bound}
This gives the range bound of a function. If $f(n) = \Theta(g(n)$, the time complexity of the function ($f(n)$) lies between $g(n)$.\\
\textbf{Rule}: $f(n) = \Theta(g(n))$ iff $\exists c_1,c_2, x_0$, such that, $c_1 g(x) \leq f(x) \leq c_2 g(x), x \geq x_0$\\
\textbf{Another way to write the rule}: $f(n) = \Theta(g(n))$ iff $f(n) = O(g(n))$ and $f(n) = \Omega(g(n))$\\
\textbf{Implication}: The growth rate of $f(n) ==$ the growth rate of $g(n)$\\
\textbf{Example}: $f(2n^3) = \Theta(g(n^3))$


\section{Recursive Fibonacci}
Code [Python]:
\begin{lstlisting}[language=python]
    def fib(n):
      if n==0 or n==1: return 1
      return fib(n-1)+fib(n-2)
\end{lstlisting}
Time Complexity:\\
$T(n) = T(n-2) + T(n-1) + O(1)$\\
Let $T(n) = r^n$\\
$\implies r^n = r^{n-1} + r^{n-2} + k$ [k:constant]\\
$\implies r^2 = r + 1 + \frac{k}{r^{n-2}}$\\
$\implies r^2 - r - c = 0$ $[c = 1 + \frac{k}{r^{n-2}}]$\\
$\implies r = \frac{1 \pm \sqrt{1 + 4c}}{2}$\\
$\implies r^n = (\frac{1 \pm \sqrt{1 + 4c}}{2})^n$\\
$\implies T(n) = r^n = (\frac{1 \pm \sqrt{1 + 4c}}{2})^n$\\
$\implies T(n) = O(k^n)$


\end{document}